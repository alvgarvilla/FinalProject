
%Siempre ponemos esto al principio, nombramos el capitulo con chapter{nombre} y entre parentesis el nombre del capitulo
\chapter{Introduction}
\label{chapter:intro}

The original aim of the Internet was to provide an open and scalable network among research and educational communities, where billions of users are served through a global system of interconnected computer networks.

\par

Unfortunately, with the rapid growth of the Internet over the last two decades, the number of attacks on the Internet has also increased rapidly. One of this attacks, consist in disrupt the service provided by a network or server, either crashing the system sending some packets that exploit a software vulnerability or sending a large number of useless traffic to collapse the resources of the service. This kind of attack is known as Denial of Service (DoS) attack, or Distributed Denial of Service attack if is launched for multiple hosts.

\par

There are some design principles of the Internet that facility these kinds of attacks~\cite{peng2007survey}:

\par

\textit{Resource sharing}: in IP networks, doubt to packet-switched service, users shares all the resources, and one user's service can be disturbed by other user's behaviour, so bandwidth attacks can disrupt service for legitimate users.
\par
\textit{Simple Core and Complex Edge}: One of the principles of Internet is that the core network should be simple and push all the complexity into the end hosts. That means that the core of the networks is not able to integrate complex application, as authentication, security. Due to this simplification, when an attacker sends packets into the network and the victim receive them, it is almost impossible recognize the real owner of the packets.
\par
\textit{Fast Core Networks and Slow Edge Networks}: The Core Networks needs to have a high capacity due to the heavy traffic that has to support from many sources to many destinations. In contrast, an edge network needs less capacity because it only needs to support its end users. A disadvantage is that traffic from high-capacity core can crush the slow-capacity edge.

\bigskip

Taking advantage of these principles and their vulnerabilities, have been arising a large number of different DoS and DDoS attacks and, as a result, a parallel growing of defense mechanisms to avoid these attacks. We might consider it like a constant battle between both sides, in which technological improvements are taking an important role on it. 
 
\par

In the current network architecture, the network devices (particular routers) are bundle with a specialized control plane and various features. This vertical integration essentially binds you to whatever software and features are shipped with those particular devices. Software Defined Networking effectively breaks these pieces apart.

\par

SDN is a type of network architecture that separates the network data plane (network devices that forwarding traffic) from the control plane (software logic that controls ultimately how traffic is flowing through the network). OpenFlow~\cite{OpenFlowWP} is a standard interface defined between the control and forwarding layers of an SDN structure. 

\par

One of the reasons to separate the control plane and the data plane is that the software control of the network can evolve independently of the hardware. 

\par

A second reason is it allows the network to be controlled from a single high-level software program. The software used to control the network (in our case, POX), even though taking in count that use a high-level programming language, has a lower layer abstraction and increase the difficulty for the Network Programmers. Due this inconvenient, it is time to speak about \textit{Frenetic}.\textit{Frenetic} is a Network Programming language which gives a high-level abstraction from POX, allowing them direct control over the network. \textit{Pyretic (Python + Frenetic)} is one of the Frenetic family programming languages which provide a domain specific sub-language for specifying dataplane packet processing.

\par

The aim of this survey is how might SDN help us to improve current DDoS defense mechanism. Throughout this project, we will review the main DDoS defense and attack mechanisms and further some algorithms already developed and how could be improve them with OpenFlow. We will test these algorithms on virtual scenarios-through Mininet.

\bigskip

This thesis is structure as follows: The next chapter we will explain the background related with this survey. We talk about the current situation of DDoS attacks and defenses and how OpenFlow works and its structure. In the chapter 3, 



